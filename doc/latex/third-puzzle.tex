\section{Third puzzle description}
As a third puzzle it was chosen \emph{round robin tournament} scheduling problem \footnote{http://en.wikipedia.org/wiki/Round-robin\_tournament}.
Round robin tournament is a competition in which all contestants meet all other contestants in a match.

Let $n$ be number of contestants and assume that there is an infinite (or sufficient) number of resources used for playing a match so there can be played whatever number of matches in parallel.
 It can be shown \cite{art} that if $n$ is odd than there will be at least $n$ rounds needed to be played. 
If $n$ is even than $n-1$ rounds are needed.
(In matter of simplicity if $n$ is odd lets consider $n$ equals to $n + 1$ so new $n$ is even.)
In each round, every contestant will play against another contestant and this contestant will be every round different.

It is easy to generate candidate solution. 
We will represent each contestant with unique number.
Each row represents one round and we know, that in each row each number can occur only once.
Example of randomly generated candidate solution for $n = 4$ can be seen below:
\begin{center}
$1-4\ 2-3$\\
$4-2\ 1-3$\\
$4-1\ 2-3$\\
$1-2\ 3-4$\\
\end{center}

This candidate solution is not optimal because it does not meet all criteria of round robin tournament -- one contestant (e.g. 1) plays against other (4) multiple times.
This will be accounted by evaluation function.

Simple evaluation function (used in our application) is based on counting how many times a each contestant played against other contestants. This number is decreased by one and than all numbers are summed. In example above, result of evaluation function ($e(s)$) can be computed as below:
$$e(s)  = h(s,1,2) + h(s,1,3) + h(s,1,4)  + h(s,2,3) + h(s,2,4) + h(s,3,4)$$
where $h(s,a,b)$ is computed as follows. $h'(s,a,b)$ is number of how many times there is a match $a-b$ or $b-a$ in state $s$. 

Scaling of this problem can be done by increasing or decreasing number of contestants.

\subsection{Classic solution}
It must be mentioned that this scheduling problem can be solved in $O(n)$ \cite{wiki}. 
Since this fact, using general constraint based solver for this problem is inefficient.
However this problem can be represented as a constrained-satisfaction, local-search problem.